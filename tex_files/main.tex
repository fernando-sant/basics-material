\documentclass[11pt]{article}
\usepackage[utf8]{inputenc}	% Para caracteres en español
\usepackage{amsmath,amsthm,amsfonts,amssymb,amscd}
\usepackage{multirow,booktabs}
\usepackage[table]{xcolor}
\usepackage{fullpage}
\usepackage{lastpage}
\usepackage{enumitem}
\usepackage{fancyhdr}
\usepackage{mathrsfs}
\usepackage{wrapfig}
\usepackage{setspace}
\usepackage{calc}
\usepackage{multicol}
\usepackage{cancel}
\usepackage[retainorgcmds]{IEEEtrantools}
\usepackage[margin=3cm]{geometry}
\usepackage{amsmath}
\newlength{\tabcont}
\setlength{\parindent}{0.0in}
\setlength{\parskip}{0.05in}
\usepackage{empheq}
\usepackage{framed}
\usepackage[most]{tcolorbox}
\usepackage{xcolor}
\colorlet{shadecolor}{orange!15}
\parindent 0in
\parskip 12pt
\geometry{margin=1in, headsep=0.25in}
\theoremstyle{definition}
\newtheorem{defn}{Definition}
\newtheorem{reg}{Rule}
\newtheorem{exer}{Exercise}
\newtheorem{note}{Note}
\begin{document}
\setcounter{section}{0}
\title{Notas sobre Equacoes de Maxwell}

\thispagestyle{empty}

\begin{center}
{\LARGE \bf Maxwell's Equations - Review Notes}\\
{\large BasiCS Physics Program}\\
2022 - 2023
\end{center}
\section{Revisao de analise vetorial}
\subsection{O operador $\vec{\nabla}$}
\begin{note}
\textbf{Neste texto, ate que se mencione diferentemente, consideraremos o conjunto de coordenadas cartesianas ($\hat{x}$, $\hat{y}$, $\hat{z}$). Para as
operacoes vetoriais, utilizaremos $\cdot$ para o produto escalar e $\times$ para o produto vetorial.}
\end{note}

Como comumente apresentado nos cursos introdutorios de calculo, uma das primeiras atuacoes do operador $\vec{\nabla}$ e vista por meio do \textbf{gradiente}.
Que, supondo um escalar $A$, possui a seguinte forma:
\begin{equation}
\vec{\nabla} A = \left(\frac{\partial A}{\partial x}\hat{x}+\frac{\partial A}{\partial y}\hat{y}+\frac{\partial A}{\partial z}\hat{z}\right)
\end{equation}
Que e o gradiente de $A$ 

Isto pode ser reescrito de um modo mais interessante como:
\begin{equation}
\vec{\nabla} A = \left(\hat{x}\frac{\partial}{\partial x}+\hat{y}\frac{\partial}{\partial y}+\hat{z}\frac{\partial}{\partial z}\right)A
\end{equation}

O termo entre parentesis e chamado de ``del' e assim o denotamos como operador $\vec{\nabla}$:
\begin{equation}
\vec{\nabla} = \left(\hat{x}\frac{\partial}{\partial x}+\hat{y}\frac{\partial}{\partial y}+\hat{z}\frac{\partial}{\partial z}\right)
\end{equation}

\subsection{Divergente e Rotacional}
Nesta subsecao, considere um vetor $\vec{A} = A_{x}\hat{x}+A_{y}\hat{y}+A_{z}\hat{z}$
\begin{shaded}
\textbf{Divergente} \newline
\begin{equation}
\vec{\nabla}\cdot \vec{A} = \left(\hat{x}\frac{\partial}{\partial x}+\hat{y}\frac{\partial}{\partial y}+\hat{z}\frac{\partial}{\partial z}\right) \cdot (A_{x}\hat{x}+A_{y}\hat{y}+A_{z}\hat{z})
                          = \left(\frac{\partial A_{x}}{\partial x}+\frac{\partial A_{y}}{\partial y}+\frac{\partial A_{z}}{\partial z}\right)
\end{equation}
%Where:
%\begin{equation*}
%\begin{split}
%G = \text{Gravitational Constant} \\
%d = \text{Object's Position Relative to Moon} \\
%d_0 = \text{Earth's Center Relative to the moon}\\
%M_m = \text{Mass of the moon}
%\end{split}
%\end{equation*}
\end{shaded}

\begin{shaded}
    \textbf{Rotacional} \newline
    \begin{equation}
    \vec{\nabla}\times \vec{A} = \begin{vmatrix}
        \hat{x} & \hat{y} & \hat{z}\\ 
        \frac{\partial}{\partial x} & \frac{\partial}{\partial y} & \frac{\partial}{\partial z}\\
        A_{x} & A_{y} & A_{z} 
   \end{vmatrix}
    = \hat{x}\left(\frac{\partial A_{z}}{\partial y}-\frac{\partial A_{y}}{\partial z}\right)+\hat{y}\left(\frac{\partial A_{x}}{\partial z}-\frac{\partial A_{z}}{\partial x}\right)+\hat{z}\left(\frac{\partial A_{y}}{\partial x}-\frac{\partial A_{x}}{\partial y}\right)
    \end{equation}
    %Where:
    %\begin{equation*}
    %\begin{split}
    %G = \text{Gravitational Constant} \\
    %d = \text{Object's Position Relative to Moon} \\
    %d_0 = \text{Earth's Center Relative to the moon}\\
    %M_m = \text{Mass of the moon}
    %\end{split}
    %\end{equation*}
\end{shaded}
\subsection{O Laplaciano - $\Delta$}
O laplaciano consiste, basicamente, no operador `del', porem, no lugar das derivadas primeiras, utilizamos as derivadas segundas.
Voces podem ter visto, ao longo de seu percurso ate aqui, diversas notacoes, contudo, as mais usadas sao $\Delta$ ou $\nabla ^2$. Entretanto,
a notacao mais utilizada ao longo dos cursos da CentraleSupelec e $\Delta$ e manteremos a mesma aqui neste texto.
\subsubsection{Laplaciano de um escalar}
Considere $\phi$ uma quantidade escalar, calculemos, então, o laplaciano deste escalar:
\begin{equation}
    \Delta \phi = \vec{\nabla} \cdot \left(\vec{\nabla} \phi\right)
                = \frac{\partial^2 \phi}{\partial x^2}+\frac{\partial^2 \phi}{\partial y^2}+\frac{\partial^2 \phi}{\partial z^2}
\end{equation}
\subsubsection{Laplaciano de um vetor}
Considere $\vec{E}$ uma quantidade vetorial, tal que $\vec{E} = E_{x}\hat{x}+E_{y}\hat{y}+E_{z}\hat{z}$. O laplaciano deste vetor é, simplesmente, o vetor com os laplacianos de cada componente escalar:
\begin{equation}
    \Delta \vec{E} = \left(\begin{matrix}
                        \Delta E_{x} & \Delta E_{y} & \Delta E_{z}
                     \end{matrix}\right) = \Delta E_{x}\hat{x} + \Delta E_{y}\hat{y} + \Delta E_{z}\hat{z}
\end{equation}
\begin{shaded}
\textbf{Important relations}\newline
Consider the following four vectors: $\vec{A}$, $\vec{B}$, $\vec{C}$ and $\vec{E}$. Then one has:
\begin{equation}
\vec{A} \times \left(\vec{B} \times \vec{C}\right) = \left(\vec{A} \cdot \vec{C}\right)\vec{B} - \left(\vec{A} \cdot \vec{B}\right)\vec{C}
\end{equation}
\begin{equation}
\vec{\nabla} \times \left(\vec{\nabla} \times \vec{E}\right) = \vec{\nabla}\left(\vec{\nabla} \cdot \vec{E}\right) - \Delta \vec{E}
\end{equation}
\begin{note}
\textbf{To make the relations easier to read, sometimes, we may use $\vec{grad} \phi = \vec{\nabla}\phi$, $\vec{div} \vec{E}= \vec{\nabla} \cdot \vec{E}$ and $\vec{rot} \vec{E}= \vec{\nabla} \times \vec{E}$, where $\phi$ is a scalar quantity.}
\end{note}
\end{shaded}
\end{document}