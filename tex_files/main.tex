\documentclass[11pt]{article}
\usepackage[utf8]{inputenc}	% Para caracteres en español
\usepackage{amsmath,amsthm,amsfonts,amssymb,amscd}
\usepackage{multirow,booktabs}
\usepackage[table]{xcolor}
\usepackage{fullpage}
\usepackage{lastpage}
\usepackage{enumitem}
\usepackage{fancyhdr}
\usepackage{mathrsfs}
\usepackage{wrapfig}
\usepackage{setspace}
\usepackage{calc}
\usepackage{multicol}
\usepackage{cancel}
\usepackage[retainorgcmds]{IEEEtrantools}
\usepackage[margin=3cm]{geometry}
\usepackage{amsmath}
\newlength{\tabcont}
\setlength{\parindent}{0.0in}
\setlength{\parskip}{0.05in}
\usepackage{empheq}
\usepackage{framed}
\usepackage[most]{tcolorbox}
\usepackage{xcolor}
\colorlet{shadecolor}{orange!15}
\parindent 0in
\parskip 12pt
\geometry{margin=1in, headsep=0.25in}
\theoremstyle{definition}
\newtheorem{defn}{Definition}
\newtheorem{reg}{Rule}
\newtheorem{exer}{Exercise}
\newtheorem{note}{Note}
\begin{document}
\setcounter{section}{0}
\title{Notas sobre Equacoes de Maxwell}

\thispagestyle{empty}

\begin{center}
{\LARGE \bf Notas sobre Equacoes de Maxwell}\\
{\large BasiCS Physics Program}\\
2022 - 2023
\end{center}
\section{Revisao de analise vetorial}
\subsection{O operador $\vec{\nabla}$}
\begin{note}
\textbf{Neste texto, ate que se mencione diferentemente, consideraremos o conjunto de coordenadas cartesianas ($\hat{x}$, $\hat{y}$, $\hat{z}$). Para as
operacoes vetoriais, utilizaremos $\cdot$ para o produto escalar e $\times$ para o produto vetorial.}
\end{note}

Como comumente apresentado nos cursos introdutorios de calculo, uma das primeiras atuacoes do operador $\vec{\nabla}$ e vista por meio do \textbf{gradiente}.
Que, supondo um escalar $A$, possui a seguinte forma:
\begin{equation}
\vec{\nabla} A = \left(\frac{\partial A}{\partial x}\hat{x}+\frac{\partial A}{\partial y}\hat{y}+\frac{\partial A}{\partial z}\hat{z}\right)
\end{equation}
Que e o gradiente de $A$ 

Isto pode ser reescrito de um modo mais interessante como:
\begin{equation}
\vec{\nabla} A = \left(\hat{x}\frac{\partial}{\partial x}+\hat{y}\frac{\partial}{\partial y}+\hat{z}\frac{\partial}{\partial z}\right)A
\end{equation}

O termo entre parentesis e chamado de ``del' e assim o denotamos como operador $\vec{\nabla}$:
\begin{equation}
\vec{\nabla} = \left(\hat{x}\frac{\partial}{\partial x}+\hat{y}\frac{\partial}{\partial y}+\hat{z}\frac{\partial}{\partial z}\right)
\end{equation}

\subsection{Divergente e Rotacional}
Nesta subsecao, considere um vetor $\vec{A} = A_{x}\hat{x}+A_{y}\hat{y}+A_{z}\hat{z}$
\begin{shaded}
\textbf{Divergente} \newline
\begin{equation}
\vec{\nabla}\cdot \vec{A} = \left(\hat{x}\frac{\partial}{\partial x}+\hat{y}\frac{\partial}{\partial y}+\hat{z}\frac{\partial}{\partial z}\right) \cdot (A_{x}\hat{x}+A_{y}\hat{y}+A_{z}\hat{z})
                          = \left(\frac{\partial A_{x}}{\partial x}+\frac{\partial A_{y}}{\partial y}+\frac{\partial A_{z}}{\partial z}\right)
\end{equation}
%Where:
%\begin{equation*}
%\begin{split}
%G = \text{Gravitational Constant} \\
%d = \text{Object's Position Relative to Moon} \\
%d_0 = \text{Earth's Center Relative to the moon}\\
%M_m = \text{Mass of the moon}
%\end{split}
%\end{equation*}
\end{shaded}

\begin{shaded}
    \textbf{Rotacional} \newline
    \begin{equation}
    \vec{\nabla}\times \vec{A} = \begin{vmatrix}
        \hat{x} & \hat{y} & \hat{z}\\ 
        \frac{\partial}{\partial x} & \frac{\partial}{\partial y} & \frac{\partial}{\partial z}\\
        A_{x} & A_{y} & A_{z} 
   \end{vmatrix}
    = \hat{x}\left(\frac{\partial A_{z}}{\partial y}-\frac{\partial A_{y}}{\partial z}\right)+\hat{y}\left(\frac{\partial A_{x}}{\partial z}-\frac{\partial A_{z}}{\partial x}\right)+\hat{z}\left(\frac{\partial A_{y}}{\partial x}-\frac{\partial A_{x}}{\partial y}\right)
    \end{equation}
    %Where:
    %\begin{equation*}
    %\begin{split}
    %G = \text{Gravitational Constant} \\
    %d = \text{Object's Position Relative to Moon} \\
    %d_0 = \text{Earth's Center Relative to the moon}\\
    %M_m = \text{Mass of the moon}
    %\end{split}
    %\end{equation*}
\end{shaded}
\subsection{O Laplaciano - $\vec{\Delta}$}
O laplaciano consiste, basicamente, no operador `del', porem, no lugar das derivadas primeiras, utilizamos as derivadas segundas.
Voces podem ter visto, ao longo de seu percurso ate aqui, diversas notacoes, contudo, as mais usadas sao $\vec{\Delta}$ ou $\vec{\nabla ^2}$. Entretanto,
a notacao mais utilizada ao longo dos cursos da CentraleSupelec e $\vec{\Delta}$ e manteremos a mesma aqui neste texto.
\begin{defn}
\textbf{Euler's Theorem} - The most general motion of any body relative to a fixed point \textit{O} is a rotation about some axis through \textit{O} To specify this rotation about a given point O, we only have to give the direction of the axis and the rate of rotation, or angular velocity $\omega$. Because this has a magnitude and direction, it is an obvious choice to write this rotation vector as $\omega$, the angular velocity vector. That is:
\begin{equation}
\omega = \omega\textbf{u}
\end{equation}
Where \textbf{u} is the unit vector
\end{defn}
\begin{shaded}
\textbf{Vector Velocity}\newline
The velocity at any point, \textit{P} (position, \textit{r}) is given by:
\begin{equation}
v = \omega\  x \ r
\end{equation}
\end{shaded}
\subsection*{Addition of Angular Velocities}
One can add angular velocities just like linear velocities. If body 3 is rotating at angular velocity $\omega_{32}$ relative to frame 2, and frame 2 is rotating at angular velocity $\omega_{21}$ relative to frame 1, then body 3 is rotating relative to frame 1 at angular velocity: 
\begin{equation}
\omega_{31} = \omega_{32} + \omega_{21}
\end{equation}
\subsection{Time Derivatives in Rotating Frames}
If frame S has a angular velocity, $\Omega$ relative to S$_0$ then the time derivative of a single vector \textbf{Q} as seen in the two frames are related by:
\begin{equation}
(\frac{d\textbf{Q}}{dt})_{S_0} = (\frac{d\textbf{Q}}{dt})_{S} \ + \Omega \ x \ \textbf{Q}
\end{equation}
\subsection{Netwon's Second Law in a Rotating Frame}
A particle in an inertial reference frame, S$_0$ obeys Newton's second law as we are use to:
\begin{equation}
m\frac{d^2r}{dt^2} = F
\end{equation}
Using the results from equation 8, the time derivative for a rotating frame with reference to an inertial frame can be given by:
\begin{equation}
(\frac{dr}{dt})_{S_0} = (\frac{dr}{dt})_s \ + \Omega \ x \ r
\end{equation}
By differentiation, Newton's second law becomes:
\begin{equation}
m\ddot{r} = F + 2m\dot{r} \ x \ \Omega \ + m(\Omega \ x \ r) \ x \ \Omega
\end{equation}
Where \textit{F} is the sum of all forces in the inertial reference frame. 
\subsection{The Centrifugal Force}
This is an inertial force in a rotating reference frame 
\begin{equation}
F_{\text{cf}} = m(\Omega \ x \ r) \ x \ \Omega
\end{equation}
\subsubsection*{Free-Fall Acceleration (Non-Vertical Gravity)}
\begin{equation}
F_{\text{eff}} = F_{\text{grav}} + F_{\text{cf}} = mg_0 + m\Omega^2R\sin(\theta)\hat{\rho}
\end{equation}
The acceleration due to the Centrifugal force is simply 
\begin{equation}
\begin{split}
g = g_0 + \Omega^2R\sin(\theta)\hat{\rho} \\
g_{\text{rad}} = g_0 - \Omega^2R\sin^2(\theta)  \\
g_{\text{tan}} = \Omega^2R\sin(\theta)\cos(\theta)
\end{split}
\end{equation}
The angle between g and its radial direction is:
\begin{equation}
\alpha \approx \frac{g_{\text{tan}}}{g_{\text{rad}}} 
\end{equation}
The maxium value at ($\theta$ = 45):
\begin{equation}
\alpha_{\text{max}} =  \frac{\Omega^2R}{2g_0}
\end{equation}
\subsection{Coriolis Force}
The Coriolis Force is another inertial force in a rotating reference frame that an object experiences when it is moving. 
\begin{equation}
F_{\text{cor}} = 2m\dot{r} \ x \ \Omega = 2mv \ x \ \Omega
\end{equation}
The maximum acceleration, \textit{a} that the Coriolis force could produce acting by itself with \textit{v} perpendicular to $\Omega$ is:
\begin{equation}
a_{\text{max}} = 2v\Omega 
\end{equation}
\begin{shaded}
\textbf{Direction of the Coriolis Force} \newline
The Direction of the Coriolis force us always perpendicular to the velocity of the object (hence equation 17), and is given by the right hand rule. 
\end{shaded}
\newpage
\subsection{Free Fall and the Coriolis Force}
\begin{equation}
m\ddot{r} = mg_0 + F_{\text{cf}} + F_{\text{cor}} 
\end{equation}
\subsection{The Foucault Pendulum}
See chapter 9, Page 354. There is no need to recopy what is in the book here. 

\end{document}