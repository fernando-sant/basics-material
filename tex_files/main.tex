\documentclass[11pt]{article}
\usepackage[utf8]{inputenc}	% Para caracteres en español
\usepackage{amsmath,amsthm,amsfonts,amssymb,amscd}
\usepackage{multirow,booktabs}
\usepackage[table]{xcolor}
\usepackage{fullpage}
\usepackage{lastpage}
\usepackage{enumitem}
\usepackage{fancyhdr}
\usepackage[geometry]{ifsym}
\newcommand{\bigexclaim}{\raisebox{-0.1em}{\BigTriangleUp}\hspace{-0.32em}\llap{\small\textbf{!}}\hspace{0.32em}}
\usepackage{pxfonts}
\usepackage{dutchcal}
\usepackage{mathrsfs}
\usepackage{wrapfig}
\usepackage{setspace}
\usepackage{calc}
\usepackage{multicol}
\usepackage{cancel}
\usepackage[retainorgcmds]{IEEEtrantools}
\usepackage[margin=3cm]{geometry}
\usepackage{amsmath}
\newlength{\tabcont}
\setlength{\parindent}{0.0in}
\setlength{\parskip}{0.05in}
\usepackage{empheq}
\usepackage{framed}
\usepackage[most]{tcolorbox}
\usepackage{xcolor}
\colorlet{shadecolor}{orange!15}
\parindent 0in
\parskip 12pt
\geometry{margin=1in, headsep=0.25in}
\theoremstyle{definition}
\newtheorem{defn}{Definition}
\newtheorem{reg}{Rule}
\newtheorem{exer}{Exercise}
\newtheorem{note}{Note}
\begin{document}
\setcounter{section}{0}
\title{Notas sobre Equacoes de Maxwell}

\thispagestyle{empty}

\begin{center}
{\LARGE \bf Maxwell's Equations - Review Notes}\\
{\large BasiCS Physics Program}\\
2022 - 2023
\end{center}
\section{Review of vector analysis}
\subsection{The operator $\vec{\nabla}$}
\begin{note}
\textbf{In this text, until otherwise mentioned, we will consider the set of Cartesian coordinates ($\hat{x}$, $\hat{y}$, $\hat{z}$). For the
vector operations, we will use $\cdot$ for the scalar product and $\times$ for the vector product.}
\end{note}

As commonly presented in introductory calculus courses, one of the first actions of the operator $\vec{\nabla}$ is seen by means of the \textbf{gradient}.
Which, assuming a scalar $A$, has the following form:
\begin{equation}
\vec{\nabla} A = \left(\frac{\partial A}{\partial x}\hat{x}+\frac{\partial A}{\partial y}\hat{y}+\frac{\partial A}{\partial z}\hat{z}\right)
\end{equation}
Which is the gradient of $A$ 

This can be rewritten in a more interesting way as:
\begin{equation}
\vec{\nabla} A = \left(\hat{x}\frac{\partial}{\partial x}+\hat{y}\frac{\partial}{\partial y}+\hat{z}\frac{\partial}{\partial z}\right)A
\end{equation}

The term in parentheses is called `del' and we denote it as the $\vec{nabla}$ operator:
\begin{equation}
\vec{\nabla} = \left(\hat{x}\frac{\partial}{\partial x}+\hat{y}\frac{\partial}{\partial y}+\hat{z}\frac{\partial}{\partial z}\right)
\end{equation}

\subsection{Divergence and Curl}
In this subsection, consider a vector $\vec{A} = A_{x}\hat{x}+A_{y}\hat{y}+A_{z}\hat{z}$
\begin{shaded}
\textbf{Divergence} \newline
\begin{equation}
\vec{\nabla}\cdot \vec{A} = \left(\hat{x}\frac{\partial}{\partial x}+\hat{y}\frac{\partial}{\partial y}+\hat{z}\frac{\partial}{\partial z}\right) \cdot (A_{x}\hat{x}+A_{y}\hat{y}+A_{z}\hat{z})
                          = \left(\frac{\partial A_{x}}{\partial x}+\frac{\partial A_{y}}{\partial y}+\frac{\partial A_{z}}{\partial z}\right)
\end{equation}
\end{shaded}

\begin{shaded}
    \textbf{Curl} \newline
    \begin{equation}
    \vec{\nabla}\times \vec{A} = \begin{vmatrix}
        \hat{x} & \hat{y} & \hat{z}\\ 
        \frac{\partial}{\partial x} & \frac{\partial}{\partial y} & \frac{\partial}{\partial z}\\
        A_{x} & A_{y} & A_{z} 
   \end{vmatrix}
    = \hat{x}\left(\frac{\partial A_{z}}{\partial y}-\frac{\partial A_{y}}{\partial z}\right)+\hat{y}\left(\frac{\partial A_{x}}{\partial z}-\frac{\partial A_{z}}{\partial x}\right)+\hat{z}\left(\frac{\partial A_{y}}{\partial x}-\frac{\partial A_{x}}{\partial y}\right)
    \end{equation}
\end{shaded}

\subsection{The Laplacian - $\Delta$}
The Laplacian consists basically of the operator `del', but instead of the first derivatives, we use the second derivatives.
You may have seen, along the way, several notations, however, the most used are $\Delta$ or $\nabla^2$. However,
the notation most often used throughout the CentraleSupélec courses is $\Delta$ and we will keep it the same here in this text.
\subsubsection{Laplacian of a scalar}
Consider $\phi$ a scalar quantity, evaluate, then, the Laplacian of this scalar:
\begin{equation}
    \Delta \phi = \vec{\nabla} \cdot \left(\vec{\nabla} \phi\right)
                = \frac{\partial^2 \phi}{\partial x^2}+\frac{\partial^2 \phi}{\partial y^2}+\frac{\partial^2 \phi}{\partial z^2}
\end{equation}
\subsubsection{Laplacian of a vector}
Consider $\vec{E}$ a vector quantity, such that $\vec{E} = E_{x}\hat{x}+E_{y}\hat{y}+E_{z}\hat{z}$. The Laplacian of this vector is, simply, the vector composed of the laplacian of each scalar component:
\begin{equation}
    \Delta \vec{E} = \left(\begin{matrix}
                        \Delta E_{x} & \Delta E_{y} & \Delta E_{z}
                     \end{matrix}\right) = \Delta E_{x}\hat{x} + \Delta E_{y}\hat{y} + \Delta E_{z}\hat{z}
\end{equation}
\begin{shaded}
\textbf{Important relations}\newline
Consider the following four vectors: $\vec{A}$, $\vec{B}$, $\vec{C}$ and $\vec{E}$. Then one has:
\begin{equation}
\vec{A} \times \left(\vec{B} \times \vec{C}\right) = \left(\vec{A} \cdot \vec{C}\right)\vec{B} - \left(\vec{A} \cdot \vec{B}\right)\vec{C}
\end{equation}
\begin{equation}
\vec{\nabla} \times \left(\vec{\nabla} \times \vec{E}\right) = \vec{\nabla}\left(\vec{\nabla} \cdot \vec{E}\right) - \Delta \vec{E}
\end{equation}
\begin{note}
\textbf{To make the relations easier to read, sometimes, we may use $\vec{grad} \phi = \vec{\nabla}\phi$, $div \vec{E}= \vec{\nabla} \cdot \vec{E}$ and $\vec{rot} \vec{E}= \vec{\nabla} \times \vec{E}$, where $\phi$ is a scalar quantity.}
\end{note}
\end{shaded}
\newpage
\section{Maxwell's Equations}
\begin{note}
    \textbf{The intention of this text is not to act as a textbook, but only to introduce relations from the world of electrodynamics,
    known as Maxwell's equations. Therefore, if you want to go further, it is valid to consult the material: Griffiths reference}
\end{note}
\subsection{Microscopic Maxwell's equations in the vacuum}
A series of experiments conducted during the nineteenth century, especially by Gauss, Faraday and Maxwell, resulted in the following set of equations which,
in turn, describe the background of the electromagnetic theory that will be used during our initial studies at CentraleSupélec. The equations are:
\begin{shaded}
    \textbf{Maxwell's Equations}\newline
    Considering the electric field as $\vec{E}\left(\vec{r},t\right)$ and the magnetic field as $\vec{B}\left(\vec{r},t\right)$, where $\vec{r}$ is the position vector and $t$ indicates the time. Then one has:
    \begin{equation}
    \vec{\nabla} \cdot \vec{E} = \frac{1}{\epsilon_0}\rho \ \text{(Gauss' law)}
    \end{equation}
    \begin{equation}
        \vec{\nabla} \cdot \vec{B} = 0 \ \text{(Gauss' law of magnetism)}
    \end{equation}
    \begin{equation}
        \vec{\nabla} \times \vec{E} = -\frac{\partial \vec{B}}{\partial t} \ \text{(Faraday's law)}
    \end{equation}
    \begin{equation}
        \vec{\nabla} \times \vec{B} = \mu_0 \vec{J} + \mu_o \epsilon_0 \frac{\partial \vec{E}}{\partial t} \ \text{(Ampère - Maxwell's equation)}
    \end{equation}
$\mu_0 = 4\pi \times 10^{-7}$H/m indicates the magnetic permeability of vacuum, $\epsilon_0 = \frac{1}{36\pi}\times 10^{-9}$F/m its electric permittivity and $\mu_0 \epsilon_0 c^2 = 1$, where $c = 3 \times 10^{8}$m/s is the speed of light in the vacuum.
\begin{note}
    \textbf{As you may know, $\vec{J}$ is the current density vector, then, sometimes, we may indicate the term $\vec{J_{d}} = \epsilon_0 \frac{\partial \vec{E}}{\partial t}$ at Ampère-Maxwell's equation as a `displacement current' to rewrite the equation as
    $\vec{\nabla} \times \vec{B} = \mu_0 \vec{J} + \mu_0 \vec{J_{d}} = \mu_0 \left(\vec{J} + \vec{J_{d}}\right)$}
\end{note}
\end{shaded}

\subsection{How to deal with these equations?}
\subsubsection{Obtaining $\vec{B}$ knowing $\vec{E}$}
\begin{enumerate}[label=(\roman*)]
    \item Write $\vec{\nabla} \times \vec{E} = -\frac{\partial \vec{B}}{\partial t}$
    \item Use the relation: $\vec{B}(\vec{r},t) = \vec{B}(\vec{r},t_0) + \int_{t_0}^{t} \frac{\partial \vec{B}}{\partial t'}dt'$
    \item Finally, verify the Gauss' law of magnetism: $\vec{\nabla} \cdot \vec{B} = 0$
\end{enumerate}

\subsubsection{Obtaining $\vec{E}$ knowing $\vec{B}$ and $\vec{J}$}
\begin{enumerate}[label=(\roman*)]
    \item Write $\vec{\nabla} \times \vec{B} = \mu_0 \vec{J} + \mu_o \epsilon_0 \frac{\partial \vec{E}}{\partial t}$
    \item Use the relation: $\vec{E}(\vec{r},t) = \vec{E}(\vec{r},t_0) + \int_{t_0}^{t} \frac{\partial \vec{E}}{\partial t'}dt'$
    \item Finally, verify the Gauss' law: $\vec{\nabla} \cdot \vec{E} = \frac{1}{\epsilon_{0}}\rho$
\end{enumerate}

\begin{shaded}
\textbf{Important concepts}

    \textbf{Lorentz's force}
    \begin{equation}
        \vec{F} = q\left(\vec{E}+\vec{v} \times \vec{B}\right)
    \end{equation}

    \textbf{Volume current density vector}
    
    If the medium contains N families of charged elements, each family characterized at $\vec{r}$ and $t$ by its particular
    density $n_{i}\left(\vec{r},t\right)$ $[m^{-3}]$ and its group speed $\vec{v_{i}}(\vec{r},t)$ $[m\cdot s^{-1}]$ considering its particules of charge $q_{i}$.
    Then, we define the volume current density vector $\vec{J}$ as:
    \begin{equation}
        \vec{J} = \sum_{i=1}^{N} n_{i}q_{i}\vec{v_{i}}\ \text{in units of $[A\cdot m^{-2}]$}
    \end{equation}

    \textbf{Local Ohm's law}
    \begin{equation}
        \vec{J} = \sigma \vec{E}\ \text{(inside an ohmic medium)}
    \end{equation}
    Where $\sigma$ stands for the medium's electric condutivity, in units of $\Omega^{-1} \cdot m^{-1}$ or even of $S \cdot m^{-1}$
\end{shaded}

\subsection{Local charge conservation}
First, we remind you of a fundamental relation in vector analysis: $div\left(\vec{rot}\vec{B}\right) = 0$ also written as $\vec{\nabla} \cdot \left(\vec{\nabla} \times \vec{B}\right) = 0$

After that, we know from section 2.1 the Ampère-Maxwell's equation: $\vec{\nabla} \times \vec{B} = \mu_0 \vec{J} + \mu_o \epsilon_0 \frac{\partial \vec{E}}{\partial t}$

Using the previous relation, we apply the divergence operator into Ampère-Maxwell:

\begin{equation}
    \vec{\nabla} \cdot \left(\vec{\nabla} \times \vec{B}\right) = \mu_{0}\left(\vec{\nabla} \cdot \vec{J}\right) + \mu_{0} \epsilon_{0} \left(\vec{\nabla} \cdot \frac{\partial \vec{E}}{\partial t}\right)
\end{equation}

Now, $\vec{\nabla} \cdot \left(\frac{\partial \vec{E}}{\partial t}\right) = \frac{\partial}{\partial t}\left(\vec{\nabla} \cdot \vec{E}\right)$

And, using Gauss' law, we have:
\begin{equation}
    \vec{\nabla} \cdot \left(\frac{\partial \vec{E}}{\partial t}\right) = \frac{1}{\epsilon_{0}}\frac{\partial \rho}{\partial t}
\end{equation}

Finally, using the previous relations, we arrive at the:
\begin{shaded}
    \textbf{Local charge conservation equation}
    \begin{equation}
        \vec{\nabla} \cdot \vec{J} + \frac{\partial \rho}{\partial t} = 0
    \end{equation}
\end{shaded}

\newpage
\section{Poynting's Theorem}
\begin{note}
    \textbf{Consider that we are still working in the free space - or vacuum}
\end{note}
As you may have seen during the previous sections or even during your studies at you home university, it is possible to associate
an energy density (per unit volume) to the electric ($\vec{E}$) and magnetic ($\vec{B}$) fields, defined as (consider $E$ and $B$ as the absolute value of the electric $\vec{E}$ and $\vec{B}$ fields, respectively):
\begin{shaded}
    \textbf{Electric energy density}
    \begin{equation}
        \mu_{E} = \frac{\epsilon_{0}}{2}E^2
    \end{equation}

    \textbf{Magnetic energy density}
    \begin{equation}
        \mu_{B} = \frac{1}{2\mu_{0}}B^2
    \end{equation}
\end{shaded}

Then, from $(20)$ and $(21)$, we can easily infer that the total energy that can be stored in the electromagnetic field, per unit volume, is given by:
\begin{shaded}
    \textbf{Electromagnetic energy density}
    \begin{equation}
        w = \mu_{E} + \mu_{B} = \frac{\epsilon_{0}}{2}E^2 + \frac{1}{2\mu_{0}}B^2
    \end{equation}
\end{shaded}

The question we may ask ourselves is: Given a charge distribution, how much work ($dP$) in a time interval ($dt$) is done by electromagnetic forces acting on them?

To try to answer this question, first, we know that, by the definition of current and charge densities in a differential volume: $q \rightarrow \rho dv$ and $\vec{J} \rightarrow \rho \vec{v}$. On the otherhand,
 the Lorentz's force expression gives us:
\begin{equation}
    \vec{F} \cdot d\vec{l} = q\left(\vec{E}+\vec{v}\times \vec{B}\right) \cdot \vec{v}dt = q\vec{E} \cdot \vec{v}dt
\end{equation}

Therefore, considering the total volume $V$ (where the charges are distributed), we arrive at:
\begin{equation}
    \frac{dP}{dt} = \int_{V}\vec{J} \cdot \vec{E}dv
\end{equation}

Then, a new definition arrives:
\begin{shaded}
    \textbf{Power density, by unit volume or Volume power}
    \begin{equation}
        \frac{\partial P}{\partial t} = \vec{J} \cdot \vec{E}
    \end{equation}
\end{shaded}

Still using the Lorentz's force expression and that $q \rightarrow \rho dv$ and $\vec{J} \rightarrow \rho \vec{v}$, we can achieve another important expression, let's see:
\begin{equation}
    \vec{F} = q\left(\vec{E}+\vec{v} \times \vec{B}\right) = \left(\rho\left(\vec{r},t\right)\vec{E}\left(\vec{r},t\right) + \vec{J}\left(\vec{r},t\right) \times \vec{B}\left(\vec{r},t\right)\right)dv
\end{equation}
Thus, we get the following definition, per unit volume:
\begin{shaded}
    \textbf{Electromagnetic force density}
    \begin{equation}
        \vec{f}\left(\vec{r},t\right) = \rho\left(\vec{r},t\right)\vec{E}\left(\vec{r},t\right) + \vec{J}\left(\vec{r},t\right) \times \vec{B}\left(\vec{r},t\right)
    \end{equation}
So, the electromagnetic force acting on the charges inside a differential volume $dv$ around the position $\vec{r}$ is $d\vec{F} = \vec{f}\left(\vec{r},t\right)dv$
\end{shaded}

Based on the concepts defined above, we can also define an energy current density vector, that is, a vector $\vec{\Pi}\left(\vec{r},t\right)$ such that
the electromagnetic energy that passes through an infinitesimal surface $\vec{dS}$ in a time interval $dt$ around the position $\vec{r}$ is:
\begin{equation}
    d^2W = \vec{\Pi}\left(\vec{r},t\right) \cdot \vec{dS}dt
\end{equation}
Where $dW = wdv_{r}$ is the amount of electromagnetic energy contained in an infinitesimal volume $dv_{r}$

Thus, we will define this energy current density vector $\vec{\Pi}\left(\vec{r},t\right)$ as:
\begin{shaded}
    \textbf{Poynting's Vector}
    \begin{equation}
        \vec{\Pi}\left(\vec{r},t\right) = \frac{1}{\mu_{0}}\vec{E}\left(\vec{r},t\right) \times \vec{B}\left(\vec{r},t\right)
    \end{equation}
\end{shaded}

Also, we can analyse the variation of energy inside a volume delimited by a surface, after some reasonings and calculations, one can derive an equation
describing the local electromagnetic energy balance, also called local Poynting's equation (for a derivation of this equation, see the next subsection):
\begin{shaded}
    \textbf{Local Poynting's theorem - Electromagnetic energy balance}
    \begin{equation}
        \frac{\partial w}{\partial t}\left(\vec{r},t\right) + \vec{\nabla} \cdot \vec{\Pi}\left(\vec{r},t\right) = -\left(\vec{J} \cdot \vec{E}\right)\left(\vec{r},t\right)
    \end{equation}
    What do these terms mean?\newline
    \newline
    The first term express the variation of how much energy relays inside the considered volume, a `storage'' term;\newline
    The second term corresponds to the energy that is transfered or displaced;\newline
    The final one corresponds to the energy that is consumed (if the term is negative) or producted inside this volume.
\end{shaded}
\subsection{Extra: Obtaining the local Poynting's equation}

Given a volume $\mathcal{V}$ delimited by a surface $\mathcal{S}$, the total electromagnetic energy contained inside this volume in a time $t$ can vary over the time.
That is because some amount of energy is lost inside the volume $\mathcal{V}$ (let's use the indicator `loss') and some amount can `escape' across the surface $\mathcal{S}$ (let's use the indicator `out')  
\newline
\newline
\bigexclaim \ Following the main vector analysis theorems and their convention, consider the surface $\mathcal{S}$ outer-oriented
\newline
\newline
Easily, calling the total electromagnetic energy as $W$ (and that $\mathcal{P}$ stands for power) one can verify that the global power balance considering the volume $\mathcal{V}$ is given by (remember the outer-oriented surface):
\begin{equation}
    \frac{dW}{dt} = -\mathcal{P}_{loss}-\mathcal{P}_{out}
\end{equation}
Then, we can calculate each of these three terms

First, the total electromagnetic energy $W$:
\begin{equation}
    W\left(t\right) = \iiint_{\mathcal{V}} w\left(\vec{r},t\right)\;dv
\end{equation}

Then, the power lost inside the volume $\mathcal{V}$, transfered to the charges inside it:
\begin{equation}
    \mathcal{P}_{loss} = \iiint_{\mathcal{V}} \left(\vec{J} \cdot \vec{E}\right)\left(\vec{r},t\right)\;dv
\end{equation}

Finally, the power going out the surface $\mathcal{S}$:
\begin{equation}
    \mathcal{P}_{out} = \oiint_{\mathcal{S}} \vec{\Pi} \cdot \vec{dS} = \Phi_{\Pi}
\end{equation}
That is the the flux of the Poynting's Vector through the surface $\mathcal{S}$

Inserting the three terms calculated inside the global power balance, we arrive at:
\begin{shaded}
    \textbf{Global electromagnetic energy balance inside a surface $\mathcal{S}$}
    \begin{equation}
        \frac{d}{dt}\left(\iiint_{\mathcal{V}} w\left(\vec{r},t\right)\;dv\right) + \oiint_{\mathcal{S}} \vec{\Pi} \cdot \vec{dS} = - \iiint_{\mathcal{V}} \left(\vec{J} \cdot \vec{E}\right)\left(\vec{r},t\right)\;dv
    \end{equation}
\end{shaded}

Using the divergence theorem (a.k.a Ostrogradsky's theorem), the second term can be replaced;
\begin{equation}
    \oiint_{\mathcal{S}} \vec{\Pi} \cdot \vec{dS} = \iiint_{\mathcal{V}} \vec{\nabla} \cdot \vec{\Pi}\left(\vec{r},t\right)\;dv
\end{equation}

So that,
\begin{equation}
    \frac{d}{dt}\left(\iiint_{\mathcal{V}} w\left(\vec{r},t\right)\;dv\right) + \iiint_{\mathcal{V}} \vec{\nabla} \cdot \vec{\Pi}\left(\vec{r},t\right)\;dv = - \iiint_{\mathcal{V}} \left(\vec{J} \cdot \vec{E}\right)\left(\vec{r},t\right)\;dv
\end{equation}

And, as this equation should be valid for all volume $\mathcal{V}$, we achieve the Local Poynting's Equation:
\begin{equation}
    \frac{\partial w}{\partial t}\left(\vec{r},t\right) + \vec{\nabla} \cdot \vec{\Pi}\left(\vec{r},t\right) = - \left(\vec{J} \cdot \vec{E}\right)\left(\vec{r},t\right)
\end{equation}
\newline
\newline
\bigexclaim \ \textbf{It is not} necessary to know this equation by heart, but understand what term means and understand the reasoning
\end{document}